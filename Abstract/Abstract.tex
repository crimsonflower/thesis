This thesis presents results of the study on determining regions sensitive to the electroweak production of supersymmetric particles in the events with two opposite-charge leptons (electrons and muons) and missing transverse momentum. 
In particular, the production of the lightest-chargino that decays through an intermediate slepton into a lepton and the lightest-neutralino was investigated.
Three models of this hypothetical decay were considered, differing in the mass-splitting pattern between the chargino and the neuralino. The research used 3.2 \invfb \, of data gathered from collisions at $\sqrt{s}$=13 TeV during run-II of the LHC in 2015. The significance values at the integrated luminosity of 3.2 \invfb \, do not provide sufficiently high sensitivity to signal. However, at an increased integrated luminosity, the sensitivity for some regions improves to the values that are significant enough to warrant searches for electroweak supersymmetry there.
