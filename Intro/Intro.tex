High Energy Physics (HEP) is a branch of physics that studies elementary particles. The main way of studying physics experimentally at subatomic level is using particle accelerators. Specific particles are accelerated to high speeds and directed to collide with each other producing an array of other particles that are then registered by detectors and studied by physicists. The largest currently functioning HEP experiment is the Large Hadron Collider (LHC) at CERN in Geneva, Switzerland. Since 2015, it is able to accelerate protons and create collisions at a centre-of-mass energy of 13 TeV. These collisions happen at two sites where the main particle detectors are situated. These experiments are ATLAS (A Toroidal LHC Apparatus) and CMS (Compact Muon Solenoid). Such high energy conditions of collisions emulate the state of the universe very shortly after the Big Bang and thus allow creation of particles that don't exist at normal conditions. 

The underlying theory of modern HEP is the Standard Model (SM) and it has been very successful in describing subatomic particles and the interactions between them. It is a very robust theory that has been tested through a plethora of experiments at colliders. The most prominent success of previous LHC runs at 7 and 8 TeV centre-of-mass energy was the discovery of the Higgs particle, which 
was the final SM particle whose existence was proven experimentally \citep{Aad:2012tfa,chatrchyan2012observation}.   

However, there are limitations to SM and there are a number of theories that try address these shortcomings. Collectively they are known as “Beyond-the-Standard-Model” (BSM) theories and present new theoretical frameworks aiming to explain physical world at a deeper level. One of the major BSM theories is supersymmetry (SUSY). It has a mathematically effective and elegant representation and, most importantly, overcomes some critical SM limitations.

Searches for evidence of supersymmetry are performed at the LHC along with other particle experiments. After runs at 7 and 8 TeV no significant evidence was found and limits on masses of hypothetical supersymmetric particles have been placed. Higher energy collisions present a better chance of finding  evidence of supersymmetry. The research presented in this thesis focuses on trying to design search methods in run-II (13 TeV) data that are sensitive to particular SUSY scenarios that feature electroweak production and decay. For that purpose a set of kinematic and topological variables are investigated as ways to determine presence of new particles.
