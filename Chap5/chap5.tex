The search for regions sensitive for specific electroweak SUSY production scenarios was performed in this thesis. The analysis was performed using the 3.2 \invfb of proton-proton collision data  gathered at $\sqrt{s}=13$ TeV in the run-II of the LHC with the ATLAS detector. Various cut strategies were investigated and sensitivities of particular regions to electroweak SUSY production scenarios were analysed. 

The previous chapter showed some combinations of cuts that can be used in searches for electroweak SUSY production. For the SF channel the $Z$ veto and no-jets requirement were the starting point and further investigation included cuts on \metrel \, and \mttwo. For the DF events a veto was placed on $b$-tagged jets and it was followed by various cuts on \metrel \, and \mttwo. 

The 400-200 GeV mass splitting model provided best values for significance across all combination of cuts. The biggest value was 6.86 for the DF \mttwo \, distribution  with \metrel$>$80 GeV and \mttwo$>$120 GeV at 19.2 \invfb of projected integrated luminosity.
The sensitivity for the 600-100 model was lower. The best achieved significance value was 4.85 for the the DF \mttwo \, distribution  with \mttwo$>$140 GeV at 19.2 \invfb of projected integrated luminosity. Finding a sensitive region for the 200-150 signal model proved challenging and no appreciably high significance value was obtained.

These results are in accordance with the theoretical assumptions. Smaller mass charginos have a higher cross section, but their signal resides in regions which are very heavily background-dominated. Higher mass charginos have a larger presence in the tails of distributions but possess much smaller cross sections and therefore produce fewer events. Thus, the fact that the intermediate signal showed the best sensitivity is not surprising. It strikes a balance between having a large enough cross section and extending into regions which are not that heavily populated with background events.

The significance values at the current gathered data luminosity of 3.2 \invfb \, show that there is little sensitivity to the SUSY processes discussed in this thesis. However, at the projected luminosity value of 19.2 \invfb there is a marked improvement in significance, especially in the DF channel. With the LHC continuing its run at 13 TeV more data will become available and this will undoubtedly improve the sensitivity to the SUSY scenarios. 







